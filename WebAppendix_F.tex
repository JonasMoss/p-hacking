\section*{Web Appendix F}

Recall that a density $f(x;\theta)$ is identifiable if $f(x;\theta_{1})=f(x;\theta_{2})$
for all $x$ implies that $\theta_{1}=\theta_{2}.$ Call a density
$f(x;\theta)$ \textit{strongly identifiable} if $f(x;\theta_{1})/f(x;\theta_{2})$
being constant for all $x$ in an open interval $I$ implies that
$\theta_{1}=\theta_{2}$.
\paragraph{Proposition A.}\label{prop:identifiable}

Let $f(x;\theta)$ be a family of densities on $\mathbb{R}$, $-\infty=a_{1}<a_{2}<\ldots<a_{k+1}=\infty$
a sequence of cutoffs, and $f_{[a_{i},a_{i+1})}(x;\theta)$ the density
$f$ truncated to $[a_{i},a_{i+1})$. Let $\lambda_{i},k=1,\ldots k$
be positive numbers satisfying $\sum_{i=1}^{k}\lambda_{i}=1$. Then
the mixture
\[
g(x;\lambda,\theta)=\sum_{i=1}^{k}\lambda_{i}f_{[a_{i},a_{i+1})}(x;\theta)
\]
is identifiable in $(\lambda,\theta)$ if $f(x;\theta)$ is strongly
identifiable in $\theta$.

\paragraph{Proof.}
Assume that $g(x;\lambda_{1},\theta_{1})=g(x;\lambda_{2},\theta_{2})$.
Then $$\lambda_{1i}f_{[a_{i},a_{i+1})}(x;\theta_{1})=\lambda_{2i}f_{[a_{i},a_{i+1})}(x;\theta_{2})$$
for all $i$, thus
\[
\frac{\lambda_{1i}}{\lambda_{2i}}=\frac{f_{[a_{i},a_{i+1})}(x;\theta_{1})}{f_{[a_{i},a_{i+1})}(x;\theta_{2})}.
\]
This implies that $f(x;\theta_{1})/f(x;\theta_{2})$ is constant for
$x\in[a_{i},a_{i+1})$. But since $f(x;\theta)$ is strongly identifiable,
$\theta_{1}=\theta_{2}$, and, consequently, $\lambda_1 = \lambda_2$.


If $f(x;\theta)$ is real analytic and nowhere zero, $f(x;\theta_{1})/f(x;\theta_{2})$
is also real analytic and nowhere zero. By the Identity Theorem \citep[Corollary 1.2.6]{Krantz2002-bt}, if the ratio
$f(x;\theta_{1})/f(x;\theta_{2})$ is constant on some interval $I$,
then $f(x;\theta_{1})/f(x;\theta_{2})$ is constant everywhere, hence
$f(x;\theta_{1})=f(x;\theta_{2})$ everywhere. Thus a family of real
analytic nowhere zero densities is identifiable if and only if it is
strongly identifiable. Every exponential family of densities on the form
\[
f(x;\theta)=h(x)\exp(\eta(\theta)^{T}T(x)-A(\theta))
\]
satisfies this property, provided only that $h$ is nowhere zero real analytic
and $T$ is real analytic. In particular, the normal family satisfies
the properties.

Not every density is strongly identifiable. For instance, mixtures of uniforms are not strongly identifiable. And indeed, Proposition A fails when $f$ is a mixture of uniforms.
